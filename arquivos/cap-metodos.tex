\lipsum[1]

Citando a figura do capítulo anterior: fig.~\ref{fig:if}

\section{Uma seção qualquer}

\lipsum[2-5]

	\begin{table}
		\centering
		\setlength{\belowcaptionskip}{0.5\baselineskip}
		\caption[Legenda curta]{tabelas são trabalhosas de se construir no \LaTeX, mas existem algumas ferramentas que podem ajudar. Por exemplo, o LaTable lhe permite criar tabelas visualmente e salvá-las no formato do \LaTeX. Além dele, alguns IDE (\foreign{Integrated Development Environment}) trazem consigo um \foreign{software} próprio para isso (o Kile é um exemplo). Finalmente, é possível baixar e instalar algumas macros para Excel e OpenOffice que convertem tabelas deles para o \LaTeX.}
		\label{tab:tabela}
		
		\begin{tabular}{ccc}
			\toprule
			Camadas	&	Espessuras (\nano\metre) & Observações\\
			\midrule
			GaAs & \multicolumn{2}{c}{Substrato semi-isolante (001)}\\
			\rowcolor{gray!20}GaAs & 50 & \foreign{Buffer} \\
			$[(\text{AlAs})_5(\text{GaAs})_{10}]\times10$	&	84 &	Super-rede (SR)	\\
			\rowcolor{gray!20}GaAs & 20 & \foreign{Buffer} \\
			\color{blue} GaAs:Si & $\sim\unit{1}{MC}$ & 	
				$n_\text{Si}=\unit{4}{\times\terad\centi\rpsquare\metre}$ \\ 
			\rowcolor{gray!20}\color{blue} GaAs & 30 & Barreira anterior\\ 
			\color{blue}$\text{In}_y\text{Ga}_{1-y}\text{As}$ & 10 & QW ($y = 16,53\%$) \\ 
			\rowcolor{gray!20}\color{blue} GaAs & 7 & Barreira posterior \\
			\color{blue} GaAs & 170 & \foreign{Buffer} \\ 
			\rowcolor{gray!20}GaAs:Si & 10 & $\unit{3,0\times10^{17}}{\centi\rpcubic\metre}$ \\
			\bottomrule
		\end{tabular}
	\end{table}

\section{Mais outra seção}

\lipsum[6]

\begin{equation}\label{eq:pitagoras}
\sen^2\phi + \cos^2\phi = 1.
\end{equation}

\lipsum[7-9]